\cite{learningtocount} uses the idea of density in order to count objects. Density is the idea that
each pixel has the ability to change the total count by a certain amount by just existing within the
image. In other words, the density is the rate of change over a single pixel. The density of every
pixel forms a \textit{density map}, and the way one is to use this density map is by calculating the
integral of the density from the first pixel to the last.
% As ^^^edit2^^^ suggested to do, I put in this text to further illustrate how density works in
% terms of counting.
%%%%%%%%%%%%%%%%%%%%%%%%%%%%%%%%%%%%%%%%%%%%%%%%%%%%%%%%%%%%%%%%%%%%%%%%%%%%%%%%%%%%%%%%%%%%%%%%%%%%
%                                                                                                  %
For example, say we have an image of asteroids, and we would like to count how many are within the
image. In this scenario, the density map should contain per-pixel values that sum to one over the
pixels associated with a single asteroid.
%                                                                                                  %
%%%%%%%%%%%%%%%%%%%%%%%%%%%%%%%%%%%%%%%%%%%%%%%%%%%%%%%%%%%%%%%%%%%%%%%%%%%%%%%%%%%%%%%%%%%%%%%%%%%%

The task then is to find a model that outputs a proper density map. As this paper points out, the
main question becomes ground truth collection. They point out that when humans count, it is common
for them to tap on each instance of the object in the image. They therefore decided that \textit{dot
annotations}, where the annotator puts a dot on each of the objects, is a reasonable expectation.
However, dot annotations don't explicitly state density, just where the objects are. Therefore, the
ground truth density needs to be modeled some other way.

Naturally, they needed to pick a loss function. They define a function called the \textit{Maximum
Excess Over Subarrays} (MESA). MESA finds the subarray (over $x$ and $y$, forming a box) whose $L_1$
distance between two sets of density maps (in this case) is greatest. They give two reasons, via two
counterexamples, for this choice. One of them is that, if we take the degenerate case of MESA where
the only subarray is the whole image, then regression may be possible, but it requires that each
sample be a full image (and also eliminates the usefulness of having a density map in the first
place; they state that this is ``a direct mapping from some global image characteristics...to the
number of objects is learned''\cite{learningtocount}[§ 1.1]). This is in contrast to the other scenario that they
mention, where the summation of distances between each density value and its counterpart in the
other density map addresses the issue of accurate density maps in theory and gives you substantially
more samples to work with (as each pixel is a sample). However, it may be the case that an input
image during training may output a density map that integrates to the right value, but small
deviations between the ground truth density map and the predicted density map would cause the
pixel-wise error to dramatically increase. They point out that this is bad for training purposes as
it does not truly encode the counting error.

They settled on the ground truth density being modeled as a 2D Gaussian kernel. The Gaussian
kernel's values were that of a normal distribution, so integrating over them summed approximately to
1, the number of objects covered by the kernel. The reason why the shape of the density does not matter, as
they state, is that, in the end, they only care about the densities summing to the count of objects
in the image. However, as stated previously, training a model to be specific about the density of
each pixel allows for less training data.

For the first dataset that they discuss (involving determining the number of biological cells in an
image), each annotation was given a kernel the size of one pixel, and that pixel's value was one as
% Needed to change the "fo" that was originally here to "for"; this was mentioned by ^^^edit2^^^
a result. The second dataset for which they were counting humans, each annotation had a kernel with
diagonal covariance matrix with 4s on the diagonal, making the kernel much wider than the former
one. These sizes were chosen empirically. The regression model used in both was a simple linear one, whose
inputs were specific features, not merely the values of the pixels in the image. Overall, the method
was successful at outperforming what they deemed ``baseline approaches''\cite{learningtocount}[§ 3].