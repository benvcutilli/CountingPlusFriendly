One solution to the adversarial examples problem is a simple and intuitive one: to penalize the
gradient with respect to the image. As stated in the literature review, this was proposed by
\cite{gu2015deep} (however, we independently developed it). To see why this makes sense, consider
the fundamental situation when it comes to adversarial examples: as \cite{goodfellow2015explaining}
states, dot products in deep learning can have \textit{many} terms. If this is unavoidable (more
on that later in other proposed defenses), then all we can do is to reduce the importance of each
pixel/subpixel, which can be seen as penalizing the gradient w.r.t. the (sub)pixels.

How to do this is not necessarily straightforward. For starters, we could penalize its norm. Which
norm to use may matter, but, in our implementation of this technique, we stuck with a normal 2-norm.
The calculation of the loss comes in two steps. The first step is do normal backpropagation towards
the image in order to get the gradient with respect to the image. The norm is then calculated, and
added to the normal loss term. Backpropagation occurs from there, which requires calculating the
``second derivative'' of the gradient (we use the derivative of the norm as the the gradient is a
vector and, therefore, its gradient is not well-defined). However, our choice of activation
function(s) may be problematic if they are not smooth (for example, the
ReLU\cite{pmlr-v15-glorot11a} activation function, which we found to be
problematic)\footnote{Double-checked by Stefan Lee}. To see why let us look at the ReLU activation
function. ReLU is specified as
\[
    \begin{cases}p & \text{for }p > 0 \\ 0 & \text{otherwise}\end{cases}
\]
where $p$ is the activation potential coming into the activation function. On one side of this
piecewise function, the derivative is always $0$, and on the other side, it is always $1$. It is
plain to see, then, that the second derivative will be zero everywhere, thus providing no utility when
it comes to gradient descent. This is a flaw in the nature of its linear, piecewise behavior;
mathematically, the second derivative does \textit{approximately} exist in the form of the
derivative of the SoftPlus~\cite{NIPS2000_44968aec} function. As a result (see § \ref{detectingshapes}), the image's
second derivative is zero. This is the case if we ignore the loss function, as one thing to consider
is the loss function's effect on the second derivative. Here, we can run into some trouble in our
hypothesis. Considering that using a smooth non-linear loss function $L$ is extremely common (for
example, mean squared error), there \textit{does} exist a non-zero second derivative for it. When
applying the product rule in this situation, for neural network $f$, we get
\[
    \frac{\partial}{\partial a} \left[ \frac{\partial L}{\partial f} \cdot \frac{\partial f}{\partial a} \right]
\]
as one of the terms in the product rule. This is non-zero (as our only hope of this being zero is if
the \textit{derivative} of the network output is zero). So, it \textit{is} possible to get some
second derivative out of a ReLU-based network (or any piecewise linear network, for that matter).
However, for our experiments, smooth functions were used to be more mathematically sound and to be
more sure that we have gotten closer to exhausting ways in which this method can go wrong.

Following \cite{goodfellow2015explaining}, we backpropagate from the correct class's loss. We will
call this method \textit{Gradient Magnitude Reduction} (GMR).