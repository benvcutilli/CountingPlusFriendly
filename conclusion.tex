Representation is a very important part of neural networks (and machine learning in general). We
have put an emphasis on subitization as well as adversarial examples in order to discuss this topic,
and shown what appears to be success in representation learning via learning to count. More complex
situations need to be explored, as well as changes to and full implementation of our detection
methods to really make it work, and what we have seen with our experiments hints at this truly being
possible. We did not demonstrate a network with full subitization capabilities, but modifications to
the training and/or network may make this a reality.


As for adversarial examples, we did not have success, but we did lay out substantial justification
for the defenses that we proposed; hopefully, this analysis leads us to (a) related method(s) that
corrects for any analytical flaws in what we have proposed. Specifically, we emphasized structure
over additional loss terms, with the exception of GMR. However, even a modified form of GMR, in the
future, might benefit from structural changes, such as when any of these modifications are used in
conjunction with methods such as Elastic Sigmoid. While papers like \textit{Adversarial Logit
Pairing}~\cite{kannan2018adversarial} appear to be very close to a cure, the space of attacks seems
boundless, and this author believes that mathematical proofs are required in order to fully address
this issue. In the author's opinion, proving that iterative optimizers and special loss terms really
achieve ideal networks is difficult due to the dynamic nature of these methods and the infinite set
of points over which they are optimizing. Seeing as methods like Adversarial Logit Pairing and
adversarial training are defenses that fit within this definition, the author is somewhat skeptical
about their true effectiveness. This is because adversarial examples like what can be found in
~\cite{eykholt2018robust} focus on fooling networks relative to \text{all} human perception, and
noise bounded by $L_{\infty}$ or similar metrics clearly does not model this. Structural changes
appear easier to prove (from an adversarial example view), so they may make the most sense to pursue
further. In the end, we look into this topic because the adversarial example community inherently
focuses nearly completely on representation.

In summation, topic of representation needs substantially more resources dedicated to it. For the
author, this thesis is merely a starting point, and we plan to build upon this work. We need to
solidify the fundamentals before we can truly unleash deep learning.