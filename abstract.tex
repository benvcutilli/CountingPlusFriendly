% It was requested by ^^^iulatex^^^ that I follow ProQuest^^^^^^ requirements for the abstract, but
% I have no idea where to find those with the exception of ^^^^^^ limiting the abstract to 775
% words.
In this work, we take a look at representation learning. To do so, we perform experiments in shape
counting in the hopes that the artificial neural network learns the representation of shapes. The
network, at least in part, appears to learn some representation resembling the shape. We formulate a
way to perform shape detection using prior-research's saliency maps in a novel way, though the
algorithm has issues to be addressed at a later time. Another representation topic that we look at
is that of adversarial examples: if a person wants to reduce the classification performance of a
neural network substantially or completely, there are ways to turn normal images into ``adversarial
examples'' without a human noticing the change. We propose a number of defenses and both
analytically and intuitively state their properties that would, in theory, lead to them performing
well. However, the theory did not pan out in any substantial way, but we make some conjectures as to
why this is the case. In all, our the experiments are semi-successful and we extensively discuss how
these topics rely heavily on representation.